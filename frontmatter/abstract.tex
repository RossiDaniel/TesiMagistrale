%!TEX root = ../dissertation.tex
% the abstract

La seguente tesi è il risultato della collaborazione svolta nel periodo intercorso tra marzo e agosto 2021 tra il sottoscritto, l'Università di Padova, nella persona del Professor Aiolli, e l'azienda Estilos.\\
Nel mondo della vendita online, ma non solo, sentiamo parlare ormai sempre più spesso di sistemi di raccomandazione, in questa tesi andremo ad applicare diversi suoi approcci in un contesto non del tutto usuale, ossia quello di un e-commerce BTB di un'azienda Cliente di Estilos. 
Si è dovuto lavorare sullo storico vendite relativo il canale online in quanto l'e-commerce non prevede la raccolta di valutazioni da parte degli utenti sui prodotti.\\
Inserito all'interno di un quadro più ampio gli obiettivi del progetto prevedono la rielaborazione dello storico vendite in modo d'avere i dati in forme più classiche, ossia rating discreti su una scala comune, a cui poi applicare gli approcci più popolari al momento nell'ambito dei sistemi di raccomandazione, come il collaborative filtering e il content based filtering.\\
Più nello specifico il task che si persegue è quello di raccomandare a ciascun cliente una lista di prodotti che si ritiene possano interessarlo.\\
Ci si proponeva inoltre di trovare prodotti simili e correlati dato uno di partenza, anche utilizzando informazioni esterne. Si voleva poi vagliare approcci ibridi che permettessero di combinare informazioni che descrivono l'interesse di un cliente verso un prodotto rispetto diverse espressioni d'interesse, quali per esempio la quantità totale acquistata e la recentezza dell'acquisto. Data la natura dei dati si voleva infine provare ad affrontare il problema come un next basket recommendation, andando a considerare le fatture come sessioni d'acquisto e studiando se questo approccio funzionasse meglio dei precedenti.