%!TEX root = ../dissertation.tex
% the abstract

Il seguente documento vuole essere un report della collaborazione svoltasi tra l'Università di Padova, nella persona del Professor Aiolli, e l'azienda Estilos SRL nel periodo intercorso tra marzo e agosto 2021.\\
Lo scopo della collaborazione è stato quello di applicare un sistema di raccomandazione ad un e-commerce BTB, dove i clienti sono intermediari che poi rivendono i prodotti a clienti terzi.\\ Il tipo di prodotti venduti nell'e-commerce non segue le classiche logiche di vendita in famosi e-commerce come può essere Amazon, quindi si è reso necessario ragionare sui diversi tipi di prodotto disponibili.
Nello specifico i dati su cui abbiamo lavorato provengono da un'e-commerce di un'azienda cliente di Estilos e non sono basati, come nei classici problemi di raccomandazioni, su recensioni o valutazioni, bensì sullo storico vendite dell'azienda nel canale e-commerce.\\
Gli obiettivi del progetto prevedevano di rielaborare lo storico vendite in modo d'avere i dati in forme più usuali, a cui poi applicare gli approcci più popolari al momento nell'ambito dei sistemi di raccomandazione, come il collaborative filtering e il content based filtering. \\Più nello specifico si vuole raccomandare a ciascun cliente una lista di prodotti che si ritiene possano interessarlo.\\ Ci si proponeva inoltre di trovare prodotti simili e correlati dato uno di partenza. Infine ci si proponeva di vagliare approcci ibridi che permettessero di combinare informazioni che descrivono l'interesse di un cliente verso un prodotto rispetto diverse prospettive quali per esempio la quantità totale acquistata e la recentezza dell'acquisto.