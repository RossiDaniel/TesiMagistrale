%!TEX root = ../dissertation.tex
% the abstract

Nel seguente documento vengono riportati i risultati ottenuti dalla collaborazione tra l'Università di Padova, nella persona del Professor Aiolli, e l'azienda Estilos SRL concretizzatasi nel seguente lavoro svolto nel periodo intercorso tra marzo e agosto 2021.\\
Nel mondo della vendita online, ma non solo, sentiamo parlare ormai sempre più spesso di sistemi di raccomandazione, in questa tesi andremo ad applicare diversi suoi approcci in un contesto non del tutto usuale, ossia quello di un e-commerce BTB di un'azienda Cliente di Estilos. 
Si è dovuto lavorare sullo storico vendite relativo il canale online in quanto l'e-commerce non prevede la raccolta di valutazioni da parte degli utenti sui prodotti.\\
Gli obiettivi del progetto prevedono la rielaborazione dello storico vendite in modo d'avere i dati in forme più classiche, a cui poi applicare gli approcci più popolari al momento nell'ambito dei sistemi di raccomandazione, come il collaborative filtering e il content based filtering.\\
Più nello specifico il task che si persegue è quello di raccomandare a ciascun cliente una lista di prodotti che si ritiene possano interessarlo.\\
Ci si proponeva inoltre di trovare prodotti simili e correlati dato uno di partenza, anche utilizzando informazioni esterne. Si voleva poi vagliare approcci ibridi che permettessero di combinare informazioni che descrivono l'interesse di un cliente verso un prodotto rispetto diverse prospettive quali per esempio la quantità totale acquistata e la recentezza dell'acquisto. Data la natura dei dati si voleva infine provare ad affrontare il problema come un next basket recommendation, andando a considerare le fatture come sessioni d'acquisto e studiando se questo approccio funzionasse meglio.   






















\begin{comment}
    Il seguente documento riporta i risultati della collaborazione svoltasi tra l'Università di Padova, nella persona del Professor Aiolli, e l'azienda Estilos SRL nel periodo intercorso tra marzo e agosto 2021.\\
    Lo scopo della collaborazione è stato quello di applicare un sistema di raccomandazione ai dati estratti un e-commerce BTB, dove i clienti sono intermediari che poi rivendono i prodotti a clienti terzi.\\ Il tipo di prodotti venduti nell'e-commerce non segue le classiche logiche di vendita in famosi e-commerce come può essere Amazon, quindi si è reso necessario ragionare sui diversi tipi di prodotto disponibili.
    Nello specifico i dati su cui abbiamo lavorato provengono da un'e-commerce di un'azienda cliente di Estilos e non sono basati, come nei classici problemi di raccomandazioni, su recensioni o valutazioni, bensì sullo storico vendite dell'azienda nel canale e-commerce.\\
    \end{comment}