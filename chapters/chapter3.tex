%!TEX root = ../dissertation.tex

\hypertarget{(chap:capitolo3)}{}
\chapter{Analisi dei dati}
\section{Principali tabelle}
Come detto precedentemente i dati sono lo storico vendite dell'azienda A estratti dal modulo hybris, questi sono organizzati secondo le tabelle SAP, avremo quindi lo storico delle fatture, dove avremo una tabella per la testata della fattura, ossia la parte descrittiva dove viene riportato l'acquirente, e una tabella per le posizioni della fattura, ossia la parte dovve vengono riportati i materiali acquistati. Abbiamo inoltre due tabelle che riportato rispettivamente l'anagrafica cliente e materiali, dove possiamo trovare informazioni aggiuntive che li descrivono.\\Mi è stata inoltre fornita anche una tabella glossario che riportava per ciascuna tabella una breve spiegazione di ogni campo. \\
Tutte queste tabelle sono state fornite in formato Excel.

Quindi ricapitolando le principali tabelle disponibili sono le seguenti:
\begin{itemize}
	\item \textbf{VBAK}: testata della fattura;
	\item \textbf{VBAP}: posizioni della fattura;
	\item \textbf{MARA}: anagrafica prodotto;
	\item \textbf{KNA1}: anagrafica materiali.
\end{itemize}

Andiamo ora a vedere per ciascuna di queste tabelle i campi annessi e alcune informazioni di natura statistica:
\subsection{Tabella VBAK}
La tabella VBAK contiene la testata di circa 35000 fatture, datate dall'anno 2016 fino a maggio 2021.\\
Ciascuna riga della tabella riporta :
\begin{itemize}
	\item \textbf{VBELN}: codice univoco che identifica la fattura;
	\item \textbf{KUNNR}: codice cliente che identifica l'acquirente;
	\item \textbf{ERDAT}: data emissione fattura;
	\item \textbf{ERZET}: orario emisisone fattura;
	\item \textbf{NETWR}: importo totale fattura;
	\item \textbf{WAERK}: EUR/USD.
\end{itemize}

\subsection{Tabella VBAP}
Ciascuna riga della tabella testata fattura riporta le seguenti informazioni:
\begin{itemize}
	\item \textbf{VBELN}: codice della fattura di appartenenza;
	\item \textbf{MATNR}: codice univoco che identifica materiale;
	\item \textbf{NETPR}: prezzo netto singola unità;
	\item \textbf{KWMENG}: quantità acquistata;
	\item \textbf{NETPR}: prezzo netto;
	\item \textbf{NETWR}: spesa totale;
	\item \textbf{WAERK}: EUR/USD.
	\item \textbf{PRODH}: codice gerarchia prodotto storico; 
\end{itemize}

\subsection{Tabella KNA1}
Ciascuna riga della tabella anagrafica clienti riporta le seguenti informazioni:
\begin{itemize}
	\item \textbf{KUNNR}: codice cliente;
	\item \textbf{LAND1}: codice paese origine;
	\item \textbf{NAME1}: nome azienda cliente;
	\item \textbf{ORT01}: località; 
	\item \textbf{REGIO}: regione.
\end{itemize}

\subsection{Tabella MARA}
Ciascuna riga della tabella anagrafica materiali riporta le seguenti informazioni:
\begin{itemize}
	\item \textbf{MATNR}: codice univoco che identifica materiale;
	\item \textbf{MATKL}: codice gruppo merceologico;
	\item \textbf{GROES}: dimensioni materiali nel formato (lunghezza X larghezza X altezza);
	\item \textbf{NTGEW}: peso netto;
	\item \textbf{GEWEI}: unità di misura del peso;
	\item \textbf{VOLUM}: volume;
	\item \textbf{VOLEH}: unità di misura del volume;
	\item \textbf{LAENG}: lunghezza;
	\item \textbf{BREIT}: larghezza;
	\item \textbf{HOEHE}: altezza;
	\item \textbf{MEABM}: unità di misura lunghezza, larghezza, altezza;
	\item \textbf{PRODH}: codice gerarchia prodotto; 
	\item \textbf{MAKTX}: descrizione testuale materiale.
\end{itemize}


\section{Prodotti}
Nelle sezioni precedenti abbiamo parlato di materiali, l'utilizzo di questo termine è dovuto al fatto che la tabella rappresenta un aggregato di prodotti di natura diversa, sono catalogati come materiali i macchinari pronti per la vendita ma anche materiali usati per il loro assemblaggio oltre che ricambi e accessori.\\
Da questo momento in poi faremo riferimento a loro chiamandoli prodotti in quanto è possibile acquistarli attraverso l'e-commerce.\\
In totale nella tabella anagrafica materiali (MARA) sono presenti circa 75000 prodotti diversi, mentre i prodotti effettivamente venduti risultano essere molti meno attestandosi all'incirca verso gli 8000.\\
Abbiamo però due campi interessanti che riguardano la gerarchia prodotto (PRODH) e il gruppo merceologico (MATKL), questi due campi ci permettono di studiare la similarità dei prodotti.

\subsection{Gerarchia prodotto (PRODH)}
Il campo gerarchia prodotto (PRODH) è un campo numerico di 18 cifre utile per separare i prodotti rispetto le diverse categorie su più livelli.
Nella tabella secondaria T179 vengono definiti i livelli di gerarchia e le diverse categorie. 
Vediamoli di seguito:
\begin{itemize}
	\item \textbf{PRODH}: codice gerarchia prodotto;
	\item \textbf{STUFE}: livello gerarchia;
	\item \textbf{VTEXT}: descrizione testuale;
\end{itemize} 

Ciascun codice PRODH contenuto nella tabella T179 avrà rispettivamente il seguente numero di cifre in base al livello di gerarchia (STUFE):
\begin{itemize}
	\item \textbf{STUFE = 1}: 1° livello della gerarchia, il codice sarà di 5 cifre.
	\item \textbf{STUFE = 2}: 2° livello della gerarchia, il codice sarà di 10 cifre, dove le prime 5 identificano la categoria di 1° livello a cui appartengono mentre le restanti 5 indentificano la sotto-categoria di 2° livello.
	\item \textbf{STUFE = 3}: 3° livello della gerarchia, il codice sarà di 18 cifre, dove le prime 10 identificano la categoria di 2° livello a cui appartengono mentre le restanti 8 indentificano la sotto-categoria di 3° livello.
\end{itemize}
Ciascun prodotto sarà quindi provvisto di un codice di 18 cifre che identificherà una categoria per ogni livello.\\
Nella tabella VBAP ci sono alcune posizioni dove a parità di codice prodotto (MATNR) si hanno codici PRODH diversi, questo è dovuto al diverso momento temporale in cui sono stati acquistati. Infatti nella tabella VBAP il codice PRODH è storico, ho provveduto per semplicità ad aggiornarli tutti al codice PRODH più recente riportato nella tabella anagrafica materiali MARA.
Il numero di prodotti interessati sono circa 100 su 10000 posizioni.\\
\subsubsection{Selezione categorie di 1° livello}

\begin{minipage}[H]{0.4\textwidth}
	\scalebox{0.45}{%
	\begin{tabular}{|l|rr|l|}
		\toprule
		PRODH &    \#tot &  \#sold &                  titolo \\
		\midrule
		00010 &    28 &    0 &              Lavasciuga \\
		00020 &     0 &    0 &             Spazzatrici \\
		00030 &     0 &    0 &            Monospazzole \\
		00040 &     5 &    0 &              Aspiratori \\
		00050 &     2 &    0 &      Car washing system \\
		00090 &     2 &    0 &     Ricambi / accessori \\
		00100 &  1117 &  173 & LAVASCIUGA UOMO A TERRA \\
		00200 &   645 &  130 & LAVASCIUGA UOMO A BORDO \\
		00250 &    31 &   11 &            SANIFICATORI \\
		00300 &   405 &   92 &             SPAZZATRICI \\
		00400 &   525 &   36 &            MONOSPAZZOLE \\
		00500 &  1715 &   70 &              ASPIRATORI \\
		00600 &   334 &    6 &          WASHING SYSTEM \\
		00700 &     1 &    0 &         SISTEMI RICICLO \\
		00900 & 70441 & 7702 &     RICAMBI \& ACCESSORI \\
		00950 &    28 &    1 &               CHEMICALS \\
		09999 &   215 &    0 &                   ALTRO \\
		\bottomrule
		\end{tabular}
		
}
\end{minipage}
\begin{minipage}[H]{0.6\textwidth}
	Nella tabella vengono mostrati i codici PRODH delle categorie di 1° livello, nella colonna \#tot il numero di prodotti diversi per quella categoria, nella colonna \#sold il numero di prodotti diversi acquistati almeno una volta appartententi a quella categoria ed infine il titolo della categoria.\\
	Come possiamo vedere le prime sei categorie con titolo in minuscolo hanno pochi prodotti catalogati in MARA e nessun prodotto venduto.\\
\end{minipage}
Chiaramento il fatto che siano tutti a zero è dovuto all'aggiornamento dei codici PRODH di cui abbiamo parlato precedentemente, a prescindere da ciò il numero di posizioni che prima riportavano codici appartenenti alle categorie prese in considerazione non superava la decina, quindi non considerare queste categorie in quanto si è smesso di usarle sembra la scelta più logica.\\
La categoria ALTRO (09999) non è stata considerata in quanto riporta prodotti che non sono disponibili sull'e-commerce.
Inoltre le categorie SISTEMI RICICLO (00700) e CHEMICALS (00950), dato il basso numero di prodotti presenti in MARA e le basse vendite, si è preferito non considerarle.
\subsubsection{Overview categorie di 1° livello}

\scalebox{0.55}{%
\begin{tabular}{|l|rrrccc|l|}
\toprule
$PRODH$ &    $\#posizioni$ &  $\sum_{KWMENG}$ &  $\mathbb{E}_{KWMENG}$ &     $\mathbb{E}_{NETPR}$ &     $\sum_{NETWR}$ &     $\mathbb{E}_{NETWR}$ &      $titolo$\\
      &                  &              &              &       (€)      &      (€)       &        (€)     &            \\
\midrule
00100 &      5908 &    15461 &    2.61 &   1613.44 &   3865.97 &  22840167.61 &  LAVASCIUGA UOMO A TERRA \\
00200 &      1936 &     3219 &    1.66 &   5898.09 &   8640.59 &  16745496.87 &  LAVASCIUGA UOMO A BORDO \\
00250 &       333 &     2949 &    8.85 &    552.99 &   3772.04 &   1377422.72 &             SANIFICATORI \\
00300 &       745 &     1390 &    1.86 &   4353.24 &   5364.34 &   3996430.94 &              SPAZZATRICI \\
00400 &       389 &     1651 &    4.24 &    708.17 &   2404.47 &    940777.84 &             MONOSPAZZOLE \\
00500 &      1133 &    12984 &   11.46 &    175.75 &   1034.63 &   1172236.58 &               ASPIRATORI \\
00600 &       153 &      494 &    3.23 &    448.52 &   1338.80 &     83501.04 &           WASHING SYSTEM \\
00900 &    239740 &  1070334 &    4.46 &     26.67 &     66.61 &  15968039.56 &     RICAMBI \& ACCESSORI \\
\midrule
      &   250339  &  1108493.51 &   4.72 & 1706.86 &   3225.75 &  63124073.16 &       valori riassuntivi \\

\bottomrule
\end{tabular}}
\newline

Nella tabella per ogni categoria $PRODH$ di 1° livello possiamo vedere:
\begin{itemize}
	\item $\#posizioni$: numero di posizioni in cui compaiono prodotti di quella categoria in fattura;
	\item $\sum_{KWMENG}$: quantità totale di prodotti acquistati appartenenti a quella categoria;
	\item $\mathbb{E}_{KWMENG}$: quantità media per fattura di prodotti acquistati appartenti a quella categoria;
	\item $\mathbb{E}_{NETPR}$: prezzo medio per fattura di prodotti acquistati di quella categoria;
	\item $\sum_{NETWR}$: spesa totale per prodotti di quella categoria;
	\item $\mathbb{E}_{NETWR}$: spesa totale media per fattura di prodotti di quella categoria;
\end{itemize}
Dalla tabella possiamo vedere come la categoria RICAMBI \& ACCESSORI riporti un prezzo medio per fattura molto più basso rispetto alle altre categorie, questo è dovuto al fatto che i pezzi di ricambio ed accessori non sono macchine o sistemi da usare per fornire un servizio quanto un prodotto per riparare quanto già si possiede, possiamo vedere che in termini di posizioni l'acquisto di pezzi di ricambio copra una cospicua parte delle posizioni in fattura, oltre che valere un importante parte del fatturato per l'azienda. Le altre categorie vendono macchine e sistemi per la pulizia quindi i prezzi medi per prodotti sono molto maggiori e per i clienti finali questi prodotti rappresentano un investimento.
Da quanto detto finora si vengono a creare due macro categorie di prodotti:
\begin{itemize}
	\item \textbf{Macchine}: questa macro categoria racchiude le seguenti categorie di 1° livello: 00100, 00200, 00250, 00300, 00400, 00500, 00600;
	\item \textbf{Ricambi}: questa macro categoria invece racchiude la sola categoria 00900.
\end{itemize}

Dobbiamo dare un'ultima precisazione infine, la maggior parte delle categorie di 2° e 3° livello risultano essere di prodotti della macro categoria delle macchine, quindi la gerarchia è molto più densa orizzontalmente per le macchine rispetto che i pezzi di ricambio, infatti per le macchine le categorie risultano essere i diversi modelli di macchinari disponibili e i prodotti a catalogo di quella categoria sono le varianti dello stesso macchinario. Per i pezzi di ricambio abbiamo solo poche categorie contenitore che li raggruppano tutti.

\subsection{Gruppo merceologico (MATKL)}
Il gruppo merceologico (MATKL) non è organizzato come una gerarchia, come lo è invece la gerarchia prodotto (PRODH), bensì come un insieme di prodotti, in totale abbiamo circa 160 gruppi, dove uno di questi contiene tutti i prodotti che prima abbiamo classificato come macchine. Rispetto il codice PRODH, il gruppo merceologico è più divisivo rispetto i ricambi, questo ci può aiutare in quanto ora siamo in grado di categorizzare anche i ricambi. 

\subsection{Dimensione, volume e peso}
I campi rigurdanti dimensione, volume e peso potrebbero essere utili per ricercare una similarità tra i prodotti.\\
Le informazioni sulle dimensioni, come lunghezza, larghezza e altezza sono praticamente ridondanti nei campi GROES e (LAENG, BREIT, HOEHE) se non per alcuni prodotti dove le informazioni sono esclusive di uno dei due campi.\\
Per peso e volume abbiamo i rispettivi campi numerici e altri due campi che riportano le unità di misura, per il volume possono essere i metri cubi o i millimetri cubi, per il peso o i kilogrammi o i grammi.
La criticità riguardo queste misure sono sulla loro scarsità, infatti su 75000 prodotti avremo informazioni su volume e peso rispettivamente solo sul 20\% e 39\%, mentre sui prodotti acquistati almeno una volta sul 19\% e 5\%. 