%!TEX root = ../dissertation.tex

\hypertarget{(chap:capitolo2)}{}
\chapter{Processi e metodologie}
In questo capitolo andremo a vedere più nel dettaglio il contesto e i motivi che hanno portato all'attivazione della collaborazione nonché gli obiettivi che si volevano raggiungere. 

\section{Contesto}
L'azienda Estilos lavora ormai da molti anni nella customizzazione del famoso gestionale SAP, cerchiamo di dare un po' di contesto riguardo questo prodotto software.
Il software SAP \textit{(System Application and Product in data processing)} è un sistema di tipo ERP \textit{(Enterprise Resource Planning)}, ossia permette la gestione dei processi di businnes e delle funzioni aziendali tramite l'ausilio di un unico software modulare dove la raccolta dei dati è centralizzata.
SAP è software leader del mercato grazie a queste ed altre caratteristiche:
\begin{itemize}
	\item \textbf{Modularità}: in quanto ogni funzione dell'azienda dispone di un modulo proprio che permette di svolgere tutte le funzioni necessarie;
	\item \textbf{Integrabilità}: dato che i dati sono centralizzati, i diversi moduli sono integrati condividendo i dati tra loro;
	\item \textbf{Customizzabilità}: è possibile modificare  integrando nuove funzionalità.
\end{itemize}

SAP ERP è quindi un insieme di moduli utilizzabili anche singolarmente, quali:
\begin{itemize}
	\item \textbf{SAP ECC}: modulo per gestire contabilità, logistica e risorse umane;
	\item \textbf{SAP CRM \textit{(Customer Relationship Management)}}: modulo che si occupa di tutte le modalità di gestione delle relazioni con il cliente;
	\item \textbf{SAP SRM \textit{(Supplier Relationship Management)}}: modulo per la gestione dei rapporti coi fornitori;
	\item \textbf{SAP PLM \textit{(Product Lifecycle Management)}}: modulo per la gestione del ciclo di vita del prodotto;
	\item \textbf{SAP SCM \textit{(Supply Chain Management)}}: modulo per la gestione della catena di fornitura dal fornitore al cliente;
\end{itemize}

Come detto Estilos si occupa di realizzare customizzazioni dei moduli SAP con particolare attenzione all'applicare elementi di intelligenza artificiale ai processi aziendali, siano essi gestionali o della customer experience, questo permette alle aziende di innovarsi intraprendendo la strada della cognitive enterprise, dove l'IA applicata ai dati supporta il processo decisionale delle aziende.

\section{Introduzione al progetto}
Ciascuno di questi moduli di SAP ERP è formato da componenti che permettono di gestire specifici aspetti. Nel caso del SAP CRM, una delle più recenti componenti è la componente e-commerce hybris perfettamente integrato con il resto del sistema. Questa può essere configurato seguendo diverse ricette, per esempio ottenendo un e-commerce BTB o BTC, nel nostro caso l'e-commerce era di tipo BTB, quindi chi acquista prodotti dall'azienda A sono a loro volta aziende che li rivendereanno / noleggeranno, vengono definiti per l'appunto come dealer. Il progetto si proponeva quindi di studiare l'applicazione di un sistema di raccomandazione al suddetto e-commerce, infatti allo stato attuale viene mostrata la stessa lista di prodotti statica per ciascun cliente e per ogni prodotto nella sua pagina associata vengono mostrati una serie di prodotti complementari e simili scelti dal reparto marketing dell'azienda. Quello che si voleva fare era provare diversi approcci per cercare di consigliare a ciascun cliente una lista di prodotti che gli risultassero interessanti. Solitamente le tecniche usate dai sistemi di raccomandazione per fare predizioni sono basate su rating, ossia valutazioni implicite / esplicite da parte del cliente sui prodotti con cui ha interagito, nel caso dello storico ordine non ne abbiamo, ma abbiamo i volumi di vendita e altre informazioni che ci possono permettere di individuare quali prodotti sono potenzialmente di maggior interesse rispetto agli altri.

\section{Vincoli temporali, tecnologici e metodologici}
La collaborazione si è svolta presso l'azienda Estilos nella sua sede di Mestre (VE) nel periodo tra marzo 2021 e agosto 2021, la mia presenza fisica in azienda è stata di 3 giorni a settimana mentre i restanti 2 giorni in smart working a causa della situazione pandemica. 
Nei primi due mesi della collaborazione si è proceduto ad uno studio del contesto del progetto e delle dinamiche di vendita, nonché dell'installazione dell'e-commerce prima nella sua versione dockerizzata, poi nella sua versione standard. Una volta a disposizione si è proceduto a vedere le diverse pagine dell'e-commerce e come era diviso, infine ho approfondito nella letterature l'approccio del collaborative filtering.
Nei successivi tre mesi ho condotto un'analisi dei dati per approfondire le diverse tipologie di prodotti e le informazioni associate allo storico vendite, ho provato diverse tecniche per trasformare tali informazioni in una misura di valutazione che ci permettesse di applicare gli approcci classici dei sistemi di raccomandazione, successivamente abbiamo indagato se fosse possibile attuare un approccio content-based con le informazioni disponibili sui prodotti, infine abbiamo provato l'approccio next basket recommendation in quanto affrontava il problema rispetto un'altra prospettiva.

Durante buona parte del periodo della collaborazione ho partecipato settimanalmente ad un colloquio aziendale dove venivano trattati temi terzi rispetto quelli della collaborazione e sempre settimanalmente ho discusso i miei progressi con il team aziendale che seguiva il progetto riguardo dubbi, criticità e soluzioni.

Prima dell'inizio della collaborazione si è concordato che la durata sarebbe stata di 5 mesi e che il sesto mese sarebbe stato utilizzato per la stesura di questo documento. 

\section{Ambiente di lavoro}
\subsection{Metodi di sviluppo}
Il \glo{ciclo di vita} di un prodotto in Trans-Cel segue il \glo{modello incrementale}, formato da un periodo di concezione dell'idea, analisi della stessa, progettazione ed infine, partendo dagli obiettivi più importanti, si realizza il prodotto rilasciando periodicamente una versione dello stesso, che dovrà mostrare le nuove funzionalità sviluppate dimostrando così l'incremento fatto. In quest'ottica, lo sviluppo di un modello può essere associato ad un incremento, la cui \glo{milestone} è la ricezione dei risultati.
A sua volta lo sviluppo di ciascun modello segue il modello incrementale, che può essere visto come l'insieme di tre principali attività:
\begin{itemize}
	\item Analisi letteratura: lettura articoli accademici riportanti modelli simili o idee per lo sviluppo degli stessi.
	\item Scrittura: scrittura del modello e integrazione con il sistema grazie al framework Or-Tools.
	\item Verifica: testing massivo e verifica delle soluzioni fornite.
\end{itemize}

\subsection{Gestione di progetto}
Per quanto riguarda la gestione di progetto sono stati utilizzati alcuni strumenti descritti con maggiore dettaglio nel \hyperlink{(chap:capitolo6)}{\textbf{Capitolo 6}}. In generale per la gestione dei task da eseguire si è fatto uso di \bit{Taiga}{taiga}, uno strumento di \glo{project management}, per la gestione della comunicazione e condivisione di informazioni si è fatto uso dell'applicazione \bit{Telegram}{telegram}, per la condivisione di documentazione e articoli si è fatto uso del servizio \bit{DropBox}{dropbox}, per il versionamento si è fatto uso del servizio \bit{GitHub}{github} per la familiarità dello strumento, infine per quanto riguarda l'interfaccia con cui versionare ho utilizzato \bit{git}{git} da terminale.
	
\subsection{Linguaggio di programmazione e ambiente di sviluppo}
Per la totalità dello stage si è lavorato utilizzando \bit{Jupyter notebook}{jupyter}. Con questo strumento è stato possibile scrivere programmi in linguaggio \bit{Python}{python} in modo molto agevole. 

Questo linguaggio di programmazione è orientato agli oggetti ed interpretato dinamicamente al momento dell'esecuzione da un interprete. \bit{Python}{python} risulta veramente versatile in quanto fornisce incredibili funzionalità utilizzabili in modo semplice e intuitivo, dispone di moltissimi moduli che permettono le più svariate operazioni, inoltre è stato possibile, grazie alla libreria \bit{Boost}{boost}, esporre costrutti \bit{C++}{cpp} a \bit{Python}{python} permettendo di utilizzarli nei propri programmi, questo ha permesso di mantenere efficienza e modularità.

