\hypertarget{(chap:capitolo5)}{}
\chapter{Tecniche combinate}
Nel capitolo precedente abbiamo visto diverse tecniche di preprocessing per ottenere dei rating dalle matrici grezze.
In questo capitolo andremo invece a vedere due approcci che si sono tentati per cercare di combinare insieme rating provenienti da fonti diverse.
\section{Premesse}
Come riportato nel capito dell'analisi dei dati, le informazioni disponibili sugli item ci permettono di valutare l'interesse dello user verso gli item secondo diversi \textit{aspetti}, quali la quantità acquistata, la spesa totale, il numero di fatture in cui compaiono e la recentezza dell'ultimo acquisto, definiremo questi aspetti da ora in poi come \textit{espressioni di interesse}.
Questi \textit{aspetti} sono organizzati in matrici grezze a cui nel capitolo precedente abbiamo applicato diverse tecniche di preprocessing andando a trasformali in rating, da qui cercare di unire insieme queste espressioni di interesse sembra essere un buon modo per migliorare la qualità delle raccomandazioni finali.
I metodi combinati prendono in input le matrici grezze e vi applicano una tecnica di preprocessing del capitolo precedente.\\
Fatto questo ci sono due modi per combinarle insieme, vediamoli di seguito:

\section{Combinazione liste $TopN$}
Il primo metodo si propone di ottenere per ogni user una lista $TopN$ di item per ciascuna espressione di interesse, queste poi andranno combinate insieme attraverso l'uso del borda count, un sistema di voting basato sulla posizione.
Vediamo ora quali sono le operazioni da attuare:
\begin{enumerate}
    \item applicare la stessa tecnica di preprocessing a tutte le matrici grezze delle espressioni di interesse ottenendo le corrispettive matrici dei rating;
    \item applicare uno degli approcci del collaborating filtering alle matrici dei rating ottenendo così le liste $TopN$;
    \item combinare insieme le liste $TopN$ secondo un sistema di voting, quale il borda count, ogni item nella lista riceve uno score in base alla posizione, si sommano gli score di ciascun item e li si riordina in base a questi.
\end{enumerate}

\section{Media matrici dei rating}
Mentre il precedente metodo prevedeva di applicare il collaborative filtering separatamente a ciascuna matrice dei rating, in questo andiamo ad effettuare una loro media ottenendo così una sola matrice dei rating.\\
A questa andiamo poi ad applicare uno degli approcci del collaborative filtering e otteniamo così la lista $TopN$.