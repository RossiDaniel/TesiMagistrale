\hypertarget{(chap:conclusioni)}{}
\chapter{Conclusioni}

\section{Problemi critici risolti}
Dopo aver visto i risultati siamo in grado di dare una valutazione generale agli stessi. Per quanto riguarda l'approccio del preprocessing alle matrici grezze siamo in grado di dire che le tecniche group-based, ossia la normalizzazione min-max e la tecnica ordered-based, non hanno restituito alcun risultato positivo. Si è però notato che confrontare le triplette per categorie in base agli item può portare ad un miglioramento dei risultati.
La tecnica product-based che gioca fortemente sulla popolarità degli item dal punto di vista delle espressioni d'interesse, ha ritornato buoni risultati soprattuto rispetto la matrice grezza della recentezza.\\
L'aver generato matrici dei rating su scale diverse è stata un'idea positiva in quanto ha influenzato i risultati finali, questi sono stati più alti dove la scala era maggiore.
I risultati in generale non sono comunque soddisfacenti in termini quantitativi e sicuramente possono essere migliorati, inoltre il tipo dataset totale, unione di macchine e ricambi insieme, non ha restituito risultati migliori delle controparti separate, quindi si ritiene opportuno considerarli come dataset separati. 
Per quanto riguarda l'approccio combinato non si sono ottenuti i risultati sperati in quanto le matrici di partenza e le liste $TopN$ prodotte sono troppo simili tra loro.
L'approccio next-basket ha restituito buoni risultati ma va ricordato che il bacino di utenti si è ridotto da circa 320 a 250 user, in ogni caso i risultati per quanto riguarda il dataset tipo macchine sono molto buoni, mentre per i ricambi siamo su valori inferiori, anche se c'è da dire che rispettivamente i due dataset lavorano su circa 500 e 7500 item.
In conclusione i risultati ottenuti sul dataset macchine sono incoraggianti, non si può dire lo stesso per quello dei ricambi, ma l'idea è quella di trattarli separatamente.

\section{Questioni aperte}
Credo possa esistere una relazione tra i risultati positivi del metodo product-based, con espressione d'interesse recentezza, e quelli dati dall'approccio next-basket, in quanto entrambi sono basati sulla popolarità e sulla recentezza. Sarebbe interessante indagare in questo senso per capire eventualmente se si possano combinare i due approcci. Sarebbe possibile trovare nei dati altre espressioni d'interesse o possiamo usarle come feature per l'approccio content-based. Non abbiamo esplicitato il confronto tra item in quanto l'approccio content-based non aveva abbastanza informazioni e il modello ItemKnn, sia per risultati scarsi che per tempi di valutazione è stato accantonato. Non sono stati affrontati problemi quali cold start per nuovi user e item.

\section{Futuri sviluppi}
Sicuramente uno degli sviluppi possibili è quello di applicare la divisione secondo categorie all'approccio product-based, andando così a confrontare per popolarità item più "simili". Inoltre anche gli approcci combinati hanno usato modelli su cui non è stato effettuato il tuning dei parametri, questo potrebbe migliorare i risultati anche se le matrici rimangono comunque molto simili tra loro, invece per ovviare a ciò potremmo combinare matrici create con i diversi metodi sviluppati.