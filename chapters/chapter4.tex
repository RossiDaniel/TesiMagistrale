%!TEX root = ../dissertation.tex

\hypertarget{(chap:capitolo4)}{}
\chapter{Preprocessing storico vendite}
In questo capitolo vedremo una descrizione dettagliata di tutte le tecniche applicate allo storico vendite con l'obiettivo di raccomandare a ciascuno user gli item per lui più interessanti.
\section{Preprocessing matrici grezze}
In questa sezione andremo a vedere diverse tecniche che sono state utilizzate per trasformare le \glo{matrici grezze} user-item in matrici dei rating.\\

\subsection{Preliminari}
Definiamo l'insieme degli user $U$, l'insieme degli item $I$ e le matrici grezze user-item $RG$.
Ciascuna tecnica lavora andando a considerare le matrici $RG$ come un vettore di triplette $V = [(u, i, RG_{(u,i)} \neq 0) |  \forall (u \in U,i \in I)]$ con $RG_{(u,i)} \in \mathbb{R^+}$.\\
Facciamo inoltre riferimento a $V_c$ come il vettore delle coppie (user,item), $V_{(u,i)}$ come il valore della tripletta di user u e item i, $V_u$ il vettore dei valori delle triplette con user u e $V_i$ il vettore dei valori delle triplette con item i.\\
Ciascuna tecnica implementa una diversa funzione $f$ biettiva di trasformazione che possiamo riassumere come segue:
$$f: [(u, i, V_{(u,i)}) |  \forall (u,i) \in V_c] \rightarrow [(u, i, r \in [1,scale] | \forall (u,i) \in V_c]$$
Queste tecniche si propongono di trasformare il valore $V_{(u,i)}$ di ciascuna tripletta in un rating $r \in [1,scale]$, con $scale$ sempre dispari.\\ 
Alcune tecniche faranno riferimento ad una distribuzione dei rating uniforme discreta o gaussian-like su gruppi di elementi. Quando ci si troverà ad applicare queste distribuzioni avremo un vettore di elementi ordinati secondo un certo criterio.\\
Vediamo come vengono assegnati i rating secondo queste due distribuzioni:
\begin{itemize}
    \item uniforme discreta: divide il vettore in modo tale che ogni valore nella scala dei rating compaia lo stesso numero di volte, assegnandoli in modo crescente, dal capo alla coda del vettore;
    \item gaussian-like: si va a definire una distribuzione normale $N(0,scale/3)$, poi si generano una quantità sufficiente di numeri secondo la suddetta distribuzione. Fatto questo si convertono tutti i numeri decimali in interi, si selezionano solo gli interi nell'intervallo $[-scale/2,scale/2]$ e si traslano nell'intervallo $[1,scale]$.\\
    Infine calcoliamo la probabilità per ciascun numero intero nella scala.\\
    Per assegnare i rating al vettore non si fa altro che iterare sugli interi dell'intervallo $[1,scale]$, andando ad eseguire in sequenza le seguenti operazioni:
    \begin{enumerate}
        \item moltiplico la probabilità di quell'intero per la lunghezza del vettore;
        \item converto il valore risultante ad intero, ottenendo quindi il numero di elementi che dovranno avere quel rating;
        \item partendo dall'inizio del vettore assegno quel rating a quello specifico numero di elementi e poi una volta raggiunto l'ultimo procedo col successivo intero della scala a partire dall'elemento seguente.
    \end{enumerate}
\end{itemize}
L'assegnazione dei rating secondo tali distribuzioni è implementato da due funzioni che restituiscono un vettore di coppie, formate dall'elemento e dal rating corrispondente.

\subsection{Tecnica product-based}
La tecnica \textit{product-based} permette di analizzare gli item da un punto di vista globale. Si procede considerando gli item in termini assoluti, vediamo di seguito le operazioni per applicarlo:
\begin{enumerate}
    \item otteniamo il seguente vettore $p = [(i,\sum V_i)| \forall i \in I]$;
    \item dopo aver ordinato il vettore $p$ basandoci sul secondo termine delle coppie, conserviamo solo il primo elemento di ciascuna di esse;
    \item andiamo ad applicare la funzione uniforme discreta / gaussian-like a tale vettore, ottenendo per ogni item un rating;
    \item per ogni tripletta si va a recuperare il rating corrispondente al suo item e crea una nuova tripletta col suddetto valore.  
\end{enumerate}

Questa tecnica porta ad avere, indipendentemente dallo user, la stessa valutazione per ogni item ed è quindi molto sensibile alla popolarità dello stesso nello storico vendite.

\subsection{Tecniche group-based}
Le tecniche presenti in questa sezione permettono di dividere il vettore delle triplette $V$ in diversi gruppi, applicare separatamente a ciascuno di essi il metodo ed infine unire insieme i vettori risultanti. \\
Va rispettata la condizione che l'intersezione tra tutti i gruppi deve essere nulla.\\
Vediamo le possibili divisioni in gruppi delle triplette:
\begin{itemize}
    \item un unico gruppo con tutte le triplette;
    \item un gruppo per ogni user contenente solo le sue triplette;
    \item per ogni user e per ogni categoria un gruppo contente tutte le triplette di quello user dove gli item appartengonoù a quella categoria;
\end{itemize}

Vediamo ora i diversi metodi applicati ad un singolo gruppo.
\subsubsection{Normalizzazione Min-Max}
Una delle prime tecniche usate per il preprocessing dei dati è quella della normalizzazione min-max, per applicarla andiamo a considerare un gruppo $G \subseteq V$ e applichiamo a ciascuna tripletta la funziona min-max, in generale otteniamo il seguente vettore risultante:
$$G = [(u, i, \frac{G_{(u,i)} - min(G_r)}{max(G_r) - min(G_r)} \in [0,1]) |  \forall (u,i) \in G_{c})]$$

Ora tutti i valori delle triplette di $G$ si troveranno in un intervallo $[0,1]$, per portarlo invece nell'intervallo $[1,scale]$ dobbiamo applicare la seguente formula:
$$[(u, i, (scale -1) \cdot \frac{G_{(u,i)} - min(G_r)}{max(G_r) - min(G_r)} + 1 \in [1,scale]) |  \forall (u,i) \in G_{c})]$$

Inoltre una volta applicata la formula, oltre che tenere i rating così come sono nel dominio dei numeri reali, si è provato anche a convertirli in numeri interi, verranno chiamate rispettivamente \textit{continous} e \textit{rint}.\\
Si voleva provare in questo modo a capire se gli user avessero volumi d'acquisto diversi e se prodotti delle stesse coppie avessero logiche d'acquisto simili. 
Inoltre dobbiamo puntualizzare che se guardiamo per esempio la distribuzione della quantità totale rispetto i prodotti, noteremo che risulta assumere il comportamento di una curva discendente, quindi ci sono molti prodotti acquistati in bassa quantità e pochi in grande quantità. Applicando questo metodo, che non va a cambiare la distribuzione iniziale dei valori ma solo a scalarli, otterremo molti rating bassi.

\subsubsection{Tecnica ordered-based}
Il seguente metodo prevede di lavorare su un gruppo di triplette $G \subseteq V$ e di eseguire le seguenti operazioni:
\begin{enumerate}
    \item ordiniamo il vettore $G$ secondo il terzo valore delle triplette;
    \item applichiamo la funzione uniforme discreta / gaussian-like a tale vettore;
    \item sostituiamo al valore della tripletta quello del rating assegnatogli.
\end{enumerate}

Questa tecnica permette di confrontare le triplette attraverso l'ordinamento e consente una migliore distribuzione dei rating rispetto la normalizzazione min-max, ma è da verificare se questa ci fornisca risultati sperimentalmente migliori.

\subsection{Approccio implicito}
Tutti gli approcci che abbiamo visto producono matrici dei rating esplicite, chiaramente un tentativo sarà quello di usare una versione della matrice grezza implicita.

\section{Tecniche combinate}
Nelle sezioni precedenti abbiamo visto diverse tecniche di preprocessing per ottenere dei rating dalle matrici grezze, in questa andremo invece a descrivere due approcci utili per cercare di combinare insieme rating provenienti dalle diverse \glo{espressioni d'interesse}.
\subsection{Premesse}
Come riportato nel capitolo dell'analisi dei dati, le informazioni disponibili sugli item ci permettono di valutare l'interesse dello user verso di essi secondo diversi \textit{aspetti}, quali la quantità acquistata, la spesa totale, il numero di fatture in cui compaiono e la recentezza dell'ultimo acquisto, definiremo questi aspetti da ora in poi come \textit{espressioni di interesse}.
Queste sono organizzate in matrici grezze, a cui nella sezione precedente, abbiamo applicato diverse tecniche di preprocessing andando a trasformali in rating. L'idea di unire insieme queste \textit{espressioni di interesse} sembra essere un buon modo per migliorare la qualità delle raccomandazioni finali.
I metodi combinati prendono in input le matrici grezze e vi applicano una delle tecniche di preprocessing illustrate nella sezione precedente.\\
Ci sono due modi per combinarle insieme che appronfiremo nelle seguenti sezioni.

\subsection{Combinazione liste \textit{TopN}}
Il primo metodo si propone di ottenere per ogni user una lista $TopN$ di item per ciascuna espressione di interesse, queste poi andranno combinate insieme attraverso l'uso del borda count, un sistema di voting basato sulla posizione.
Vediamo ora quali sono le operazioni da attuare:
\begin{enumerate}
    \item applicare la stessa tecnica di preprocessing a tutte le matrici grezze delle espressioni di interesse ottenendo le corrispettive matrici dei rating;
    \item usare uno degli approcci del collaborating filtering sulle matrici dei rating ottenendo così le liste $TopN$;
    \item combinare insieme le liste $TopN$ secondo un sistema di voting, il borda count, nel quale ogni item della lista riceve uno score in base alla posizione, questi si sommano e gli item vengono riordinati in base al punteggio risultante.
\end{enumerate}

\subsection{Media matrici dei rating}
Mentre il precedente metodo prevedeva di applicare il collaborative filtering separatamente a ciascuna matrice dei rating, in questo calcoliamo una media delle matrici grezze, da cui, dopo aver eseguito una delle tecniche di preprocessing su di essa, ricaviamo una sola matrice dei rating.\\
A questa andiamo poi ad applicare uno degli approcci del collaborative filtering e così otteniamo la lista $TopN$.

\section{Approccio content-based}
In questa sezione vedremo un'applicazione dell'approccio content-based utilizzando le informazioni descrittiva disponibili sugli item.\\
Abbiamo diverse fonti d'informazioni:
\begin{itemize}
    \item categorie rispetto i diversi livelli;
    \item gruppo merceologico;
    \item nome del prodotto;
    \item dimensioni;
    \item volume;
    \item peso.
\end{itemize}

In generale ogni informazione descrittiva, può essere classificata in due modi:
\begin{itemize}
    \item Binaria: un'item può avere o meno questa caratteristica;
    \item Continua: l'item può avere o meno questa caratteristica, ma se ce l'ha l'informazione è un valori in un intervallo.
\end{itemize}

L'idea di base è di creare un vettore dove ogni posizione corrisponda ad una specifica caratteristica, nel caso l'informazione descrittiva sia binaria si va ad assegnargli una cella dove vi sarà 1 se l'item la possiede, 0 altrimenti. Nel caso questa sia continua verrà divisa in sotto-intervalli gestiti come caratteristiche binarie.

\subsection{Gerarchia prodotto}
A ciascuna categoria della gerarchia abbiamo assegnato una cella del vettore.
Come detto in precedenza, per la gerarchia prodotto abbiamo diverse categorie per ogni livello, per ciascuna di esse abbiamo assegnato una cella del vettore delle caratteristiche, se un'item apparteneva ad una di queste categorie andavamo ad impostare quella specifica cella a 1, altrimento 0.

\subsection{Gruppo merceologico}
Per il gruppo merceologico si è fatto come nel caso precedente, quindi per ogni codice merceologico si è andato ad assegnargli una cella del vettore delle caratteristiche.

\subsection{Nome prodotto}
Il nome del prodotto poteva darci informazioni aggiuntive sulla similarità tra di essi, osservando l'anagrafica materiali notiamo che i nomi sono assegnati in modo organizzato: prodotti simili in cui cambiano solo alcune componenti riportano lo stesso nome con le specifiche diverse.\\
Per ogni nome prodotto sono state eseguite le seguenti operazioni:
\begin{enumerate}
    \item eliminazione di simboli e informazioni ridondanti, per esempio alcuni nomi avevano riportata anche la dimensione, già disponibile in altri campi;
    \item separazione delle restanti parole;
    \item inserimento delle suddette in un dizionario contenente la totalità delle parole e il rispettivo numero di occorrenze.
\end{enumerate}
Dopo un'ulteriore pulizia a mano siamo andati a selezionare le parole con almeno 10 occorrenze, ottenendo alla fine circa 600 parole. Ad ognuna di esse è stata assegnata ad una cella del vettore delle caratteristiche, se un item nel suo titolo conteneva una di queste parole, si impostava a 1 la cella corrispondente.

\subsection{Dimensione, volume, peso}
Per le informazioni riguardanti lunghezza, larghezza, altezza, volume e peso si volevano creare per ciascuno di essi 20 intervalli aventi circa lo stesso numero di item. Si è assegnato al vettore delle caratteristiche una cella per ogni intervallo e se un'item aveva una delle misure all'interno dello stesso lo si impostava ad 1.

\subsection{Profilo user}
Una volta creata la matrice avente sulle righe gli item e sulle colonne le caratteristiche, si è proceduto a calcolare i profili degli user facendo la media delle righe corrispondenti agli item acquistati dallo stesso.

\section{Approccio next-basket}
In questo capitolo opereremo sullo storico vendite considerando le fatture come sessioni d'acquisto a cui applicheremo la \bit{\textit{User Popularity-based CF}}{aiolli}.\\
Ragionando in termini di logica d'acquisto è possibile che i dealer acquistino sempre lo stesso gruppo di item, questo ci porta a parlare del concetto di popolarità e di come questa possa variare nel tempo.
Dato che questa soluzione è stata provata dopo le altre precedentemente descritte, si è notato come non fosse così facile "battere" il modello basato sulla popolarità, quindi si è pensato di muoversi in questa direzione per vedere se i risultati migliorassero.\\
L'obiettivo era quello di predire per ciascuno user gli item dell'utima fattura.

\subsection{Premesse}
Definiamo l'insieme degli user $U$, l'insieme degli item $I$ e consideriamo ciascuna fattura come una transazione $b_{t}^{u}$, dove $t$ indica la posizione ordinale nell'insieme delle transazioni di uno user di cardinalità $B_{u}$ definito come $\mathcal{B}_u = \{b_{u}^{t} | t \in 1, \dots, B_u\}$. Definiamo $\mathcal{B}_{u}^{i} = \{b_{u}^{t}|b_{u}^{t} \in \mathcal{B}_{u} \wedge i \in b_{u}^{t}\}$, ossia l'insieme di tutte le transazione dello user $u$ in cui compare l'item $i$.
\subsection{Popolarità}
Possiamo calcolare la popolarità di un item rispetto ad uno user, detta \textit{popularity user-wise}, con la seguente formula: $\pi_{i}^{u} = \frac{\mathcal{B}_{u}^{i}}{\mathcal{B}_{u}}$.
Dato che la popolarità può variare nel tempo il paper introduce il concetto di recentezza, concludendo che per predirre l'ultima transazione potrebbe non essere efficace considerare quelle più vecchie, quindi attraverso una finestra temporale sulle quelle più recenti si va a calcolare la \textit{recency aware user-wise popularity}, un modo di stimare la popolarità di un item per uno user solo su un ristretto numero di transizioni, di seguito la formula: 
$$\pi_{u}^{i}@r = \frac{\sum_{t = \max (B_{u}-r,0)}^{B_{u}}[i \in b_{u}^{t}]}{\min(r,B_{u})}$$
Con il parametro $r$ si definisce la finestra di transazioni, a partire dall'ultima, da tenere in considerazione, la funzione $[i \in b_{u}^{t}]$ ritorna 1 se l'item $i$ è presente nella transazione $b_{u}^{t}$, 0 altrimenti. Se $r \geqslant B_{u}$ allora questa formula diventa equivalente alla \textit{popularity user-wise}.
\subsection{User Popularity-based CF (UP-CF)}
Questa soluzione, che si basa sul collaborative filtering, permette di trovare item interessanti per uno user andando ad osservare user simili ad esso, questa similarità tra due user $u$ e $v$ si concretizza nella funzione coseno asimmetrica: $w(u,v) = \frac{|I_u \cap I_v|}{|I_u|^{\alpha}|I_v|^{1-\alpha}}$ con $\alpha \in [0,1]$, dove il parametro $\alpha$ permette di bilanciare la probabilità $P(u|v)$ e $P(v|u)$. 
\subsubsection{Predizione}
Si vuole ora combinare la similarità tra user e il concetto di popolarità di un item per uno user, per fare ciò usiamo la formula: 
$$\hat{r}_{i}^{u} = \sum_{v \in U} w(u,v)^{q}\pi_{u}^{i}$$
Dove il termine $q$ è un parametro operante sulla località degli user, per un alto valore di $q$ consideremo solo user molto simili a quello target.