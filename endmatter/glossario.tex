%!TEX root = ../dissertation.tex

\begin{itemize} 
	\item \gloTarget{matrici grezze user-item} matrici aventi sulle righe gli user e sulle colonne gli item, nell'incrocio troviamo misurazioni quantitative, quali la quantità per esempio. Sono dette grezze in quanto devono essere raffinate per ottenere successivamente dei rating;
	\item \gloTarget{espressioni d'interesse} aspetti che cercano di misurare l'interesse di uno user verso un'item, nello specifico di questo lavoro sono la quantità acquistata, la spesa totale e il numero di fatture di quell'item, la recentezza dell'acquisto;
	\item \gloTarget{dealer} sono aziende che acquistano prodotti da un'altra azienda e poi li rivendono a clienti terzi;
	\item \gloTarget{preprocessing} serie di operazioni atte a manipolare i dati per migliorare le performance dei modelli a cui vengono applicati;
	\item \gloTarget{matrice dei rating} una matrice use-item avente valori in una scala ben definita;
	\item \gloTarget{backpropagation} algoritmo per l'allenamento delle reti neurali;
	\item \gloTarget{TopN} lista di lunghezza N contenente tutti gli item ordinati secondo gli interessi di uno user;
\end{itemize}