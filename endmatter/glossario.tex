%!TEX root = ../dissertation.tex

\begin{itemize} 
	\item \gloTarget{agile} modello di ciclo di vita nato per sopperire alla rigidità dei modelli precedenti, caratterizzato da rilasci rapidi e incrementali, usato per rispondere velocemente alle richieste dei clienti in termini di nuovi requisiti.
	\item \gloTarget{best bound} è una soluzione fornita da un modello dopo che si è raggiunto il tempo limite di esecuzione, questa soluzione non è ottima ma rappresenta il miglior risultato disponibile.
	\item \gloTarget{brainstorming} incontri di gruppo creativi utili per risolvere problemi o individuare nuove idee. Servono più di due partecipanti in modo che vi sia una discussione arbitraria e quindi utile.
	\item \gloTarget{ciclo di vita} insieme degli stati che il prodotto assume dal concepimento al ritiro.
	\item \gloTarget{euristica} algoritmo progettato per risolvere un problema velocemente, spesso una strada obbligata per risolvere problemi molto difficili.
	\item \gloTarget{GIL} meccanisco usato dagli interpreti dei linguaggi di programmazione per sincronizzare i thread in modo che ve ne sia in esecuzione in coda.
	\item \gloTarget{modello incrementale} modello di ciclo di vita che prevede rilasci multipli e successivi, dove ciascuno di essi realizza un incremento di funzionalità. I requisiti vengono trattati per importanza, prima quelli di maggior importanza in modo che possano stabilizzarsi con il rilascio delle versioni fino a quelli minori.
	\item \gloTarget{milestone} punto nel tempo al quale associamo un insieme di stati di avanzamento.
	\item \gloTarget{open source} termine utilizzato per riferirsi ad un software di cui i detentori dei diritti sullo stesso ne rendono pubblico il codice sorgente.
	\item \gloTarget{project management} gestione delle attività di analisi, progettazione, pianificazione e realizzazione degli obiettivi di un progetto, compito svolto dal project manager di un'azienda attraverso anche strumenti idonei.
	\item \gloTarget{slack} tempo aggiuntivo ad un'attività che ha lo scopo di evitare ritardi nella produzione del prodotto.
	\item \gloTarget{soluzioni} costrutto software e termine usato per indicare l'insieme di pacchi da disporre e le loro coordinate che permettono di collocarli nel contenitore ed il contenitore stesso..
	\item \gloTarget{solver} software commerciale o open source che permette di risolvere problemi di programmazione lineare.
	\item \gloTarget{stackable} termine utilizzato per indicare se un pacco può avere sopra di sé altri pacchi.
	\item \gloTarget{time limit} termine usato per indicare un tempo limite entro il quale può essere ricercata una soluzione dal solver.
	\item \gloTarget{vehicle routing} famiglia di problemi che trattano tutti gli aspetti della gestione di una flotta di veicoli nell'ambito della logistica.
\end{itemize} 
